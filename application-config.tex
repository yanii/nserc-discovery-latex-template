% !TEX root = main.tex
%% -----------------------------------------------------
%% Configuration of an NSERC application
%% Defines a few macros that are global to all documents of the application
%% ----------------------------------------------------

%% If your application involves a company, put the name of the company
%% in a macro rather than writing it directly.
\newcommand{\namecompany}{University of X}

%% Application year. Only appears in the metadata of the generated PDF.
\newcommand{\applicationyear}{20XX}

%% The list of all authors of the application. Again, only useful for the
%% PDF metadata
\newcommand{\authorlist}{Your Name}

%% The name and NSERC PIN of the main applicant
\nsercname{Your Name}
\nsercpin{NSERCPIN}
\nserctitle{NSERC Discovery Proposal: Your Title}

%% Documents are not dated
\date{}

%% Paragraphes français
%\setlength{\parindent}{0pt}

%% Hack to have list items displayed in a more compact way
\usepackage{paralist}
\setlength{\pltopsep}{4pt}
\setlength{\plitemsep}{4pt}

%% ----------
%% Loading a few packages. These are all optional and can be commented
%% out if you with. Feel free to add others.
%% ----------
% \usepackage{hyperref}
% \hypersetup{%
%   pdfauthor = {\authorlist{}},
%   pdfcreator = {LaTeX},
%   pdfsubject = {NSERC Discovery Application\applicationyear{} \namecompany{}}
% }
% \usepackage{url}
\usepackage{xcolor}
% \usepackage{todonotes}
% \usepackage{comment}

%% Useful: a few "todo" macros to display colored boxes with remarks
%% and comments
% \newcommand{\todo}[1]{\todo[inline,caption={},color=cyan]{\sf\small TODO: #1}}
% \newcommand{\todomarty}[1]{\todo[inline,caption={},color=pink]{\sf\small Marty: #1}}
% \newcommand{\todoall}[1]{\todo[inline,caption={},color=yellow]{\sf\small #1}}

%% This will print a "DRAFT" watermark on all pages.
%% Uncomment the next two lines once the application is ready.
% \usepackage{draftwatermark}
% \SetWatermarkText{DRAFT}